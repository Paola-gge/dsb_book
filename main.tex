\documentclass[12pt,final,a4paper,oneside]{book}
\usepackage{lmodern,graphicx,longtable,enumerate,tabularx,booktabs,amsmath,suffix,xspace,xcolor}
\usepackage{hyperref,booktabs}
\usepackage[ngerman,english]{babel}
\usepackage[utf8]{inputenc}
\usepackage[T1]{fontenc}
\usepackage[longnamesfirst]{natbib}

\usepackage{lipsum}
\usepackage{fancyvrb}
% This defines the chapter authors
\newcommand\chapterauthor[1]{\authortoc{#1}\printchapterauthor{#1}}
\WithSuffix\newcommand\chapterauthor*[1]{\printchapterauthor{#1}}

\makeatletter
\newcommand{\printchapterauthor}[1]{%
	{\parindent0pt\vspace*{-25pt}%
		\linespread{1.1}\large\scshape#1%
		\par\nobreak\vspace*{35pt}}
	\@afterheading%
}
\newcommand{\authortoc}[1]{%
	\addtocontents{toc}{\vskip-10pt}%
	\addtocontents{toc}{%
		\protect\contentsline{chapter}%
		{\hskip1.3em\mdseries\scshape\protect\scriptsize#1}{}{}}
	\addtocontents{toc}{\vskip5pt}%
}
\makeatother

% define environment for the abstract:
%   Generate the environment for the abstract:
\newenvironment{abstract}
{ \paragraph*{Summary} }
{ \bigskip}

\newenvironment{goals}
{ \noindent  \paragraph*{Learning Objectives} \begin{itemize}}
	{ \end{itemize}\bigskip}	


\newenvironment{testquestion}
{ \noindent   \section{Test Questions} \begin{itemize}}
	{ \end{itemize}}	

\newenvironment{glossy}
{ \noindent   \section{Glossary} \begin{description}}
	{ \end{description}}	


% Define the R-Logo
\definecolor{Rcolor}{RGB}{150,160,190}

\newcommand{\R}{%
	\raisebox{.3em}{\hspace{1.2em}%
		\llap{\resizebox{1.09em}{.5em}{\color{black}$\bigcirc$}}%
		\llap{\resizebox{1.199em}{.55em}{\color{darkgray}$\bigcirc$}}%
		\llap{\resizebox{1.19em}{.52em}{\color{gray!50}$\bigcirc$}}%
		\llap{\resizebox{1.1em}{.5em}{\color{gray}$\bigcirc$}}%
		\llap{\resizebox{1.25em}{.55em}{\color{gray}$\bigcirc$}}%
	}%
	\hspace{-.85em}%
	\textbf{%
		\textcolor{black}{\textsf{R}}%
		\hspace{-.025em}\raisebox{.01em}{\llap{\textcolor{Rcolor}{\textsf{R}}}}%
	}%
	\xspace}

\pagestyle{plain}
%%%%%%%%%%%%%%%%%%%%%%%%%%%%%%%%%%%%%%%%%%%%%%%%%%%%%%%%%%%%%%%%%

\begin{document}
	
	\title{Project Work in Data Science for Business}
	\author{Joudi Bacha 
		\and Isabella Emde 
		\and Nair Gokul 
		\and Paola Guerrero 
		\and Felix Hoffmann 
		\and Lara Kolb 
		\and Vaibhav Kulkarni 
		\and Vidit Maru 
		\and Maria Rueda 
		\and Caroline Weyers
		\and Ryan Zidago 
		}
	\date{Dr. Stephan Huber\thanks{stephan.huber@hs-fresenius.de} (Module Coordinator)\\
		\today}
	\maketitle
	\thispagestyle{empty}
	\newpage
	
	\chapter*{Preface}
	This document should help to describes and paraphrases how the project work in the course \textit{Data Science for Business} should look like. The paper is structured as follows: Chapter \ref{ch:sketch} gives an idea of how the project paper can be structured. In \autoref{ch:listoftopics} the titles of the projects are listed and \autoref{ch:procedure} offers a brief outlook.
	
	Overall, the idea of the project is that we write a textbook together that can work as an applied introduction to data science for students of business administration. Each project consists of one chapter. Of course, students can work together, form groups, and divide work with respect their preferences and talents, respectively. However, every student is responsible (and is graded) for his/her chapter and his/her presentation only. 
	
	Moreover, I will implement (and monitor) your working process using GitHub. I will explain you how that works in a video. 
	
	Please notice, that this document is not written to be self-explaining. It should rather work as a template. I explain everything in further detail during a GoToMeeting. 
	
	
	\tableofcontents
	\listoftables
	\listoffigures
	
	\chapter{A Sketch of How a Project Should Look Like}\label{ch:sketch}
	\chapterauthor{Barbara Streisand \\ barbara@email.de}
	
	\begin{abstract}
		This chapter offers a sketch on how a project may be structured. A project should include a summary, learning objectives, a motivation, a methodolocigal section, a section with an application in \R, an example from businesses, a conclusion, exercises, test questions, and a glossary.\\ 
		\textit{An abstract should make clear in simple language what the chapter is all about. Its maximum length is 200 words.} 
	\end{abstract}
	
	\begin{goals}
		\item ;weruhkltewiouthb
		\item swdfgwserg
	\end{goals}
	
	
	\section{Motivation}
	Motivate your topic. Discuss why the topic and the content of your chapter may be of importance for the reader. Make the content of your chapter subject of a discussion. Outline clearly what the reader can expect. Include a description of contents. For example, write at the end of this section ``The remainder of the chapter is structured as follows: Section 1.2 introduces the theory...''. Additionally, mention which related topics will not be discussed. Recommend literature for self-study here.\footnote{Making a footnote or citing in \LaTeX\ is easy. For example, \citet[see][p. XY]{Provost2013Data} or \citep{Provost2013Data}}
	
	\section{Methodological Issues}
	Bring the statistical theory or econometric approach to the reader. Try to raise a basic understanding of the meaning and the difficulties of the respective method. 
	
	\section{Applications in R}\label{sec:applicationsinR}
	Explain how your topic can be addressed by using \R and RStudio, respectively. In particular, introduce \R-packages that allow to implement the respective methods. Offer a \R-script that allows the reader to understand both the method and the programming skills that are needed to apply the method. It should be possible for the reader to replicate the stuff that is presented in the section and the script.
	
	\section{Example(s) From the Real World}\label{sec:examplerealworld}
	Make an impressive\footnote{The example should be a significant academic contribution, i.e., the example is published in a highly ranked academic journal. Alternatively, the economic impact of the example should be sizable.} example from the \textit{real world} to deepen the application and the methods discussed above. The example can stem either from the \textit{academic world} or the \textit{business world}. For example, pick an \textit{academic} research paper and summarize it to the reader in a way that he can understand the strengths and weaknesses of the respective investigation. Or, you can explain a business case where the method of interest plays an important role to solve a problem or to earn money, for example. 
	
	Sections \ref{sec:applicationsinR} and \ref{sec:examplerealworld} can be combined.
	
	\section{Conclusion}
	Conclude the chapter. This may include a short summary, an outlook and/or related literature.
	
	\section{Exercises}
	\begin{enumerate}[(1)]
		\item Design an exercise that helps the reader to understand and repeat important concepts.
		\begin{enumerate}[a)]
			\item Challenge the reader to think about the topic.
			\item Encourage a discussion.
			\item Bring up new aspects of the method.
			\item Be creative.
		\end{enumerate}
	\end{enumerate}
	
	
	\begin{testquestion}
		\item Ask five short test questions that refer to five important insights from your chapter. A discerning reader should have no problems with these questions.
	\end{testquestion}
	
	\begin{glossy}
		\item[word] This text should define the meaning of word.
	\end{glossy}
	
	
	\chapter{List of Topics}\label{ch:listoftopics}
	\chapterauthor{Michael Jackson\\ mj@king.com}
	
	\begin{abstract}
		Read the following titles of potential projects and send me an email with three topics  you could imagine working on (\url{stephan.huber@hs-fresenius.de}). Rank them according to your preferences. I will try to consider your wishes.  
		Feel free to make your own suggestions. But please, don't forget to name overall three (!) topics. If everybody likes to have the same topic, I flip a coin and/or try to consider your second and third preference. 
	\end{abstract}
	
	\begin{enumerate}
		
		\item{Data Collection in Data Science}
		
		\item{Exploratory Data Analysis in Data Science}
		%Data scientist can tell a story with data. Introduce a selection of methods that can help to find and \textit{tell a story} with data.
		
		\item{Regression Analysis in Data Science}
		
		\item{Spatial Analysis in Data Science}
		
		\item{Nearest Neighbor Analysis in Data Science}
		
		\item{Decision Trees Analysis in Data Science}
		
		\item{Text Mining Analysis in Data Science}
		
		\item{Big Data Analysis in Data Science}
		
		\item{Machine Learning in Data Science}
		
		\item{Single-Board Computers in Data Science}
		
	\end{enumerate}
	
	
	\chapter{Further Procedure}\label{ch:procedure}
	\chapterauthor{Bob Ross\\ goodnight@paint.org}
	
	The project work includes the following:
	\begin{itemize}
		\item Write a chapter on you topic as paraphrased above. Length of the paper should be about 25 pages.
		\item Prepare a presentation of 20 minutes length (we will see how we manage that part).
		\item Use the Latex to write the chapter.
		\item Use Github to include your chapter into the book which contains all chapters.
		\item Use R and upload your R-code to Github.
	\end{itemize}
	
	\section{\LaTeX}
	
	\LaTeX\ is easy and powerful. It allows the author to focus on the content because he don't need to take care about the layout very much. There are millions of sources online that provide tutorials. You can include figures in an floating environment\footnote{The placement of the figure is optimized automatically.}, see \autoref{fig:td}. Tables are possible, too. See \autoref{tab:tabex} or you can reference online sources, see \url{https://texdoc.net/texmf-dist/doc/latex/booktabs/booktabs.pdf} for a guide to make nice tables.
	
	\begin{figure}\centering
		\includegraphics[width=0.5\linewidth]{pic/td}\caption{This is a picture}\label{fig:td}
	\end{figure}
	
	\begin{table}\centering
		\caption{This is a table}\label{tab:tabex}
		\begin{tabular}{@{}llr@{}} \toprule\multicolumn{2}{c}{Item} \\ \cmidrule(r){1-2}Animal & Description & Price (\$)\\ \midrule Gnat  & per gram  & 13.65 \\& each      & 0.01 \\Gnu   & stuffed   & 92.50 \\Emu   & stuffed   & 33.33 \\Armadillo & frozen & 8.99 \\ \bottomrule
		\end{tabular}
	\end{table}
	
	\section{Git and GitHub}
	GitHub  provides hosting for software development version control using Git which is a version control system designed to handle projects with many contributors. Both tools are heavily used in software engineering and data science. It is particularly powerful when teams work on projects with a procedural workflow.
	
	\section{Style of Writing}
	Assume your reader is a well informed master student of business administration. Make things easy for your reader. Believe me, this sounds so easy but it is actually the most difficult task for scientists and academic writers. I got a lot out of reading  \textit{Writing Tips for Ph.  D. Students} from \cite{Cochrane2005Writing}\footnote{You can download this file using \url{https://t1p.de/br5o}}. While you are not a Ph. D. student, these writing tips apply for all authors who aim to communicate efficiently. 
	
\chapter{Topic 1}\label{ch:topic1}
	\chapterauthor{Student \\ student@email.de}
	
	\begin{abstract}
		This 
	\end{abstract}
	
	\begin{goals}
		\item ;
	\end{goals}
	
	\section{Motivation}

	\section{Methodological Issues}

	\section{Applications in R}\label{sec:applicationsinR}

	\section{Example(s) From the Real World}\label{sec:examplerealworld}

	\section{Conclusion}

	\section{Exercises}
	\begin{enumerate}[(1)]
		\item Design 
		\begin{enumerate}[a)]
			\item C
		\end{enumerate}
	\end{enumerate}
	
	
	\begin{testquestion}
		\item A
	\end{testquestion}
	
	\begin{glossy}
		\item[w] T
	\end{glossy}
	


\chapter{Data Collection}\label{ch:data collection}
\chapterauthor{Paola Guerrero \\ guerrero\_gonzalez.paola@stud.hs-fresenius.de}


\begin{abstract}
	Learning and building knowledge is a natural inclination of human beings, we are in search of learning new things every day, whether its business, marketing, humanities and many others, data play an important role. Any process that requires searching for information and knowledge, one of the first things must be to collect data. It helps us have a better understanding of fields that are unknown to us, have more depth in a specific topics, and solution of problems. Data collection facilitates and improves decision-making processes, the quality of decisions made without data to support them does not have the same impact compared to decisions that have a well-grounded basis. Companies also rely on this method since their strategies and their performance has become effective by tracking their progress and monitoring their data. 
\end{abstract}

\begin{goals}
	\item You will be able to identify the importance of Data collection in Data Science.
	\item You will be able understand the importance of an API to access and import information into R.
	\item You will be able to collect and analyze the data of twitter users using R.
	\item You will be able to list one key package in R that is used to deal with text mining.
	\item You will be able to have a better understanding of the benefits that Twitter data has on real-world application. 
	
\end{goals}

\section{Motivation}
The combination of data from multiple sources and disciplines enables the generation of new datasets, information, and knowledge \citep{JAEGER2010371}). Furthermore, the availability of open data facilitates innovation and offers opportunities to governments, businesses and entrepreneurs to harness the power of data for economic, social and scientific gains  \citep{SADIQ2017150}).
Data recompilation helps market researchers collect all the raw data and bring benefits by providing valid information, helping you achieve one a purpose. Fundamentally, people understand the importance of collecting data as it will depend on the right research being carried out. Likewise, if the right sources are not sought, the investigation will go down the wrong path.

Most modern companies use data in various ways. The most prominent would be as a means to improve their business strategy and essence in all possible ways. Technological advances has helped us collect information that can be of great benefit, using tools such as web site surveys, paper questionnaires, case studies, checklists, assisted interviewing systems, websites and many more are beneficial to gather new information.


By having a sufficient amount of data, it is possible to observe if a product or project that is being developed has potential or needs further analyzed.  Data extracted from certain apps are also used by researchers with different backgrounds (pollsters, marketers, academics from different disciplines) to answer a variety of questions, ranging from simple information about particular users or events.
In the \textbf{Methodological Issues} section the reader will have a better understanding of the approaches regarding data collection and how Twitter and social media is a powerful tool to gather data. \textbf{Application in R} section we will analyze Bill Gates's account by mining tweets in R. \textbf{Examples from the real world} section we wil analyze how data collection in Twitter could possibly predict future performances of companies.



\section{Methodological Issues}

Data is one of the most important and valuable resources businesses have in today’s market. The more information you have about your customers, the better you can adapt to their needs and interest\footnote{\href{https://www.lotame.com/what-are-the-methods-of-data-collection/}{https://www.lotame.com/what-are-the-methods-of-data-collection/}}. To have a better understanding of data collection, one has to have in mind that there are different types of data collection methods\citep{harrell2009data}):  

\begin{itemize}
	
	\item Primary data collection refers to the data collected on firsthand. Typically, it is data obtained straight from one’s audience.
	\item Second data collection refers to the data that is gathered after another party initially recorded it.
	\item Third data collection refers to information a company has collected from numerous sources.
	
\end{itemize}

One can also divide the methods into Quantitative data which comes in the form of numbers, quantities, and values or Qualitative data that is descriptive rather than numeric. By having into consideration all of these aspects, the collection of data will become less complex. To have a better understanding of this broad topic, let us consider Data collection on Twitter. 
Analyzing tweets and social interaction on Twitter can help to answer social science research questions, as it is very public and it is easy to access the messages, Contrary to Facebook and Instagram. 

Being Twitter a versatile communications platform to users around the globe, people can interact and express their opinions, Twitter is also an excellent source of information\footnote{\href{https://www.researchgate.net/publication/276974275_Data_collection_on_Twitter}{https://www.researchgate.net/publication/276974275\_Data	\_collection\_on	\_Twitter}}. Data extracted from Twitter is used by researchers from all over the world with different backgrounds, to answer different types of questions, starting from simple information about particular users or events main questions might be: How many followers does a given user have? Who is the most active user tweeting under a certain hashtag? Which users are central in a large network? It also includes questions of how the information goes viral among a group of users. Depending on the purpose, different tools can be applied by using Twitter text mining through R\footnote{\href{https://medium.com/@wanjirumaggie45/the-power-of-social-media-analytics-twitter-text-mining-using-r-1fceb26ac32b}{https://medium.com/@wanjirumaggie45/the-power-of-social-media-analytics-twitter-text-mining-using-r-1fceb26ac32b}}.

While \citep{inbook})argues that Twitter is used in many countries and languages, user communities change significantly concerning their size, position, and usage habits. Researchers should be aware of the seemingly small details that may be reflected in the data, for example, peaks in use over a day, or change in activity over the weekend compared to weekdays, this can incredibly affect the data collected. Researchers should also be aware that many active users do not tweet daily, or perhaps even weekly, while others are very active and alter the representativeness of a sample accordingly.
However, to analyze data on twitter it is necessary to obtain Twitter API, which is the acronym for Application Programming Interface\footnote{\href{ https://www.mulesoft.com/resources/api/what-is-an-api}{Link https://www.mulesoft.com/resources/api/what-is-an-api}}. It is software acting as an intermediary function that allows applications to communicate with each other. 

To collect data on Twitter it is necessary to obtain an API. rather than offering a single API, there are three data interfaces available on Twitter: REST API and Streaming API which are free, followed by the Search API which consists of deeper data analytics and requires a subscription. In a broad aspect, API allows users to access Twitter data in real-time. By using these services, one can search for tweets published in the past, stream tweets in real-time, manage Twitter accounts and ads.  Data is constantly flowing from the requested URL, and it is up to the researcher to develop or employ tools that maintain a persistent connection to this stream of data while simultaneously processing it.
On the first hand, one has to authenticate when requesting Twitter API, Twitter uses an “open protocol to allow secure authorization in a simple and standard method from web, mobile, and desktop applications. One has to answer a list of questions to establish in which way the collected data is going to be used. This is a security method to secure the platform data and their users\footnote{\url{https://medium.com/@GalarnykMichael/accessing-data-from-twitter-api-using-r-part1-b387a1c7d3e.\#cjisrtv0v}}

The issues concerning this applied method of research based on Twitter data are most commonly since not all types of data and forms of analysis are accurate and precise in answering all research questions. Due to its tendency to be based on data rather than questions, much of the current quantitative research on Twitter focuses on measuring and comparing specific parameters in data samples\footnote{\href{https://www.mzes.uni-mannheim.de/socialsciencedatalab/article/collecting-and-analyzing-twitter-using-r/\#collecting-tweets-using-the-rtweet-package}{https://www.mzes.uni-mannheim.de/socialsciencedatalab/article/collecting-and-analyzing-twitter-using-r/\#collecting-tweets-using-the-rtweet-package}}. This often leads to very large samples and lacks the well-grounded sets of specific research questions.
This is not the case with traditional instruments such as surveys and conventional content analysis, it should be noted that even the exploratory phase of research is markedly quantitative when exploring social media. Searching, filtering, and sorting are the only viable way to make masses of content readable to the human researcher, and are a logical first step in any analysis, even in qualitative studies. At the same time, quantitative research must be aligned with the questions asked. It must take into consideration how representative Twitter users are of the general population, both on Twitter and off it. 

Making judgments about Twitter user populations based on tweets alone can be unfavorable in the process of data collection, since some users read the content but are not as active in publishing and might be overlooked by researchers. Too much emphasis may be placed on very vocal users and the potential of research may be lost. Research on the general population based on Twitter must not be considered a definitive result, more research must be done outside the application to confirm a theory or for future valid data collection. 


\section{Applications in R}\label{sec:}
There are many ways to collect Twitter data, however, in this section, we will be using Rs rtweet package to analyze Bill Gates tweets, to give the reader a better understanding on how to collect Twitter data using R\footnote{\href{https://www.r-project.org/about.html}{https://www.r-project.org/about.html}}. R provides a wide variety of statistical, linear and nonlinear modeling, classical statistical tests, time-series analysis, classification, clustering, graphical techniques and many more. 
As mentioned in the last section, it is necessary to have API access to retrieve the tweets and apply for a developer account using the following website: 
https://developer.twitter.com/en/apply-for-access\footnote{\href{https://developer.twitter.com/en/apply-for-access}{https://developer.twitter.com/en/apply-for-access}}

One has to fill in a simple application form, which includes explaining in more detail what is going to be analyzed. Once the application has been accepted you must receive important credentials that will be necessary to use in R including Customer key, consumer secret, access token, access secret.

Once you have all the information from above, we are now able to start R and download the “rtweet” package that will help you extract the tweets\footnote{\href{https://towardsdatascience.com/a-guide-to-mining-and-analysing-tweets-with-r-2f56818fdd16}{https://towardsdatascience.com/a-guide-to-mining-and-analysing-tweets-with-r-2f56818fdd16}}.\\
\\
Install.packages("rtweet")\\ 
library (rtweet)\\
\\
After you have completed the step above it is necessary to place your authentication information, this is the information you requested when applying for twitter API.
\\
twitter\_token <- create\_token(
app = ****,
\\
consumer\_key = ****,
\\
consumer\_secret = ****,
\\
set\_renv = TRUE)\\
\\
Depending on the information that you are looking for, it is recommended that you search for tweets that contain a specific word or hashtag, it is also good to know that we can only extract tweets from the past 6 to 9 days. When extracting Twitter data, we want to make sure we analyze a specific user account, in this case, we will be analyzing Bill Gates and we have to implement the get\_timeline function. Have in consideration that Twitter only allows you to extract 3200 tweets. An example would be:\\
\\
Gates <- get\_timeline("@BillGates", n= 3200)\\
\\
Now we will analyze the tweets and discover the best and least performing tweets and have a clear overview of the account from Bill Gates. To do so, you have to distinguish between organic tweets that reach your followers without the use of ads or promotions, retweets, and replies\footnote{\href{https://blog.twitter.com/en_us/a/2014/introducing-organic-tweet-analytics.html}{https://blog.twitter.com/en\_us/a/2014/introducing-organic-tweet-analytics.html}}. The following example allows you to remove retweets and replies, leaving you with the organic tweets which are the ones we are interested in.


\begin{flushleft}
	\# Remove retweets
	Gates\_tweets\_organic <- Gates\_tweets[Gates\_tweets\$is\_retweet==FALSE, ]\\ 
	\# Remove replies\\
	Gates\_tweets\_organic <- subset(Gates\_tweets\_organic, is.na(Gates\_tweets\_organic\$reply\_to\_status\_id))
\\
\end{flushleft}
After the example from above, you’ll want to analyze engagement by looking at the variables: favorite\_count which refers to the number of likes or retweet\_count which refers to the number of retweets. Arrange them in descending order for example:
\\
\\
Gates\_tweets\_organic <- Gates\_tweets\_organic \%>\% arrange(-favorite\_count)
Gates\_tweets\_organic[1,5]Gates\_tweets\_organic <- Gates\_tweets\_organic \%>\% arrange(-retweet\_count)
Gates\_tweets\_organic[1,5]
\\
\\
Analyzing the replies, retweets and organic tweets can tell you a great deal about the type of account you’re analyzing. Some accounts are known to have exclusively more retweets, without any individual content. Finding a good ratio of replies, retweets, and organic help you differentiate between accounts; therefore, a key metric also helps monitor if one wishes to improve the performance of his or her account.
Make sure to create three different data sets. As you’ve already created a dataset containing only the organic tweets in the previous steps, simply now create a dataset containing only the retweets and one containing only the replies. This will help you to be more organized when you analyzing the data.
\begin{flushleft}
	\# Keeping only the retweets
	Gates\_retweets <- Gates\_tweets[Gates\_tweets\$is\_retweet==TRUE,]
	\# Keeping only the replies
	Gates\_replies <- subset(Gates\_tweets, !is.na(Gates\_tweets\$reply\_to\_status\_id))
\\
\end{flushleft}
Create a separate data frame containing the number of organic tweets, retweets, and replies. 
\begin{flushleft}
	\#Creating a data frame
	data <- data.frame(
	category=c("Organic", "Retweets", "Replies"),
	count=c(2856, 192,120)
\end{flushleft}
Once you’ve done that, you can prepare your data frame, which includes adding columns that calculate the ratios, percentages and some visualizations such as specifying the legend and rounding up your data.
\begin{flushleft}

	\# Adding columns
	data\$fraction = data\$count / sum(data\$count)
	data\$percentage = data\$count / sum(data\$count) * 100
	data\$ymax = cumsum(data\$fraction)
	data\$ymin = c(0, head(data\$ymax, n=-1))\\
	\\
	\# Rounding the data to two decimal points
	data <- round\_df(data, 2)\\
	\\
	\#Type\_of\_Tweet <- paste(data\$category, data\$percentage, "\%")
	ggplot(data, aes(ymax=ymax, ymin=ymin, xmax=4, xmin=3, fill=Type\_of\_Tweet)) +
	geom\_rect() +
	coord\_polar(theta="y") +
	xlim(c(2, 4)) +
	theme\_void() +
	theme(legend.position = "right")
\end{flushleft}

For the example above, the result is that Bill Gates has 90.15\% organic tweets, 3.79\% replies and 6.06\% retweets. By this we can come to the conclusion that Bill Gates Twitter account has mostly contentof his own.
It is also possible to have an overview of the activity that Bill Gates has in months or years. in the following example, one can analyze the frequency of his tweets.\\
\\
colnames(Gates\_tweets)[colnames(Gates\_tweets)=="screen\_name"] <- "Twitter\_Account"ts\_plot(dplyr::group\_by(Gates\_tweets, Twitter\_Account), "year") +
ggplot2::theme\_minimal() +
ggplot2::theme(plot.title = ggplot2::element\_text(face = "bold")) +
ggplot2::labs(
x = NULL, y = NULL,
title = "Frequency of Tweets from Bill Gates",
subtitle = "Tweet counts aggregated by year",
caption = "\nSource: Data collected from Twitter's REST API via rtweet"

	
The example form above will helps us see what is the frequency of tweets from Bill Gates, the Tweets aggregated by year from Bill Gates was at its highest between the years 2012 and 2014. In 2018 there is a drop in the frequency of his tweets.This next part will be analyzing which words are mostly used by Bill Gates. The line of code below provides you with a better understanding.\\
\\
tweets \%>\% # gives you a bar chart of the most frequent words found in the tweets
count(word, sort = TRUE) \%>\%
top\_n(15) \%>\%
mutate(word = reorder(word, n)) \%>\%
ggplot(aes(x = word, y = n)) +
geom\_col() +
xlab(NULL) +
coord\_flip() +
labs(y = "Count",
x = "Unique words",
title = "Most frequent words found in the tweets of Bill Gates",
subtitle = "Stop words removed from the list")

The most frequent words found in the tweets of Bill Gates was: "People" as the most used word, "world", "im", "progress", "lives", "polio", "energy", "book", "heres", "health", "fight", "read", "global", "video" and "students".Due to his work as a philanthropist\\
\\
We can also see the most common hashtags used by Bill in  his Twitter account. A good to visualize this is by using a word cloud like the following example:
\begin{flushleft}
Gates\_tweets\_organic\$hashtags <- \\as.character(Gates\_tweets\_organic\$hashtags)\\
Gates\_tweets\_organic\$hashtags <- gsub("c\\(", "",\\ Gates\_tweets\_organic\$hashtags)
set.seed(1234)
wordcloud(Gates\_tweets\_organic\$hashtags, min.freq=5, scale=c(3.5, .5), random.order=FALSE, rot.per=0.35, 
colors=brewer.pal(8, "Dark2")) 
\end{flushleft}

The most used hashtags from Bill Gates are: "Polio","Malaria","Family planning","Endmalaria","Aids","Givingtuesday","Smartaids","India","Toilet".

\section{Example(s) From the Real World}\label{sec:examplerealworld}
Recently, several researchers have proposed methods that use social media data, including Twitter data, to predict real-world events\citep{LIU20153893}).
\citep{GERBER2014115})presents a crime prediction model, based on linguistic analysis and statistical topic modeling, that uses spatiotemporally tagged tweets across Chicago, Illinois. Researchers show that their prediction methods that use social media data outperform existing predictors that forecast real-world events, such as Oscar award winners and many more. 

For example, in order to predict population health indices, \citep{NGUYEN201722})propose a mathematical model based on the distributions of textual features over Twitter data.
\citep{LIM2019100007})makes reference to a case study for United Airlines which validates whether or not collecting data is applicable to identify if Twitter users feedback have causal effects on enterprise outcomes. They analyze the United Express Flight 3411 incident, where a passenger was forcibly dragged out of a United Airlines plane. This caused Twitter users to be very upset. In relatively short period the Twitter data was collected, but among the whole dataset, 140, 286 tweets were containing the term “united airlines”, “united airline”, “unitedairlines”, (the abbreviation for “united airlines”) are extracted for this case study.
The United Express Flight 3411 incident induced Twitter users to use certain terms (e.g., “drag”, “remove”, “forcible”). There were results provided by \citep{LIM2019100007}) case study that also indicate identified influential term groups can be used to predict enterprise future outcomes. 


\section{Conclusion}
 
Throughout this chapter we have been able to understand more about data collection and the methods in which data collection in gathered, we can also observe how collecting data has changed due to technology and how researchers carry out their investigations. We have successfully carried out examples on how data collection can be used with R and how social media tools can be beneficial when collecting data for a specific research. Companies should invest more in internal infrastructure and make sure they are up to date with all technologies. Every department should be digitized, however, we can see that many start with an old mentality and the process of collecting data becomes more frustrating and tedious. Data Collection can be a short or long process depending on what the objective is, however, we can not rule out that companies need to invest more time and money in training and giving the necessary tools to employees.




\section{Exercises}

Social media has become an important part of everyone’s life, it has almost become an addiction for most, especially young people. This social media survey questions are designed to collect data and gain more knowledge on how the students from Hochschule Fresenius mange their media usage \footnote{\href{https://www.questionpro.com/survey-templates/social-media-survey/}{https://www.questionpro.com/survey-templates/social-media-survey/}}.

\begin{enumerate}[1)]
\item Considering your experience with social media websites Facebook, Twitter, Instagram, how likely are you to recommend it to your family and friends?

	\begin{enumerate}[a)]
	\item Extremely highly 
	\item Very likely	
	\item Somewhat likely 	
	\item Not at all likely 
	\end{enumerate}
	
\item Which of the following social media websites do you currently have an account?
	\begin{enumerate}[a)]
	\item Facebook
	\item Twitter
	\item Instagram
	\item LinkedIn
	\end{enumerate}

\item What is your go-to device to access your social media feed?

	\begin{enumerate}[a)]
	\item Tablet
	\item Mobile 
	\item Laptop
	\item desktop
	\end{enumerate}

\item How often do you check-in to your social media accounts in any given week?

	\begin{enumerate}[a)]
	\item Daily 
	\item Every two days
	\item Once a week 
	\item Every hour 
	\item Never 
	\end{enumerate}

\item Which of the following social media websites do you visit most frequently?

	\begin{enumerate}[a)]
	\item Instagram
	\item Facebook
	\item Twitter
	\item Snapchat
	\end{enumerate}

\item On a regular day how many times do you post pictures, comments etc. on your social media accounts?

	\begin{enumerate}[a)]
	\item Extremely often
	\item Very often
	\item Moderately often
	\item Slightly often
	\item Not at all often 
	\end{enumerate}
	
\item On an average how much time do you spend on social media?

	\begin{enumerate}[a)]
	\item Less than 30 minutes
	\item hour
	\item 1-2 hours
	\item 3-4 hours
	\item More than 4 hours
	\end{enumerate}

\item What is your purpose of using social media website?

	\begin{enumerate}[a)]
	\item To meet new people
	\item To find a date
	\item To promote your business
	\item To find employment 
	\item Event planning
	\end{enumerate}
	
\item Has social media affected your relationship with loved ones?

	\begin{enumerate}[a)]
	\item Yes
	\item No
	\end{enumerate}

\end{enumerate}
This is a practical example to give the reader an insight of what is like to collect data though a survey. This social media survey helps experts analyze how much social media affect the students of Hochshule Fresenius. With this survey there is more information that tells us in more detail what are the personalities and attributes of the students and how much time they use on social media, making it easier for researchers to study.  No other research method can provide this broad capability, which ensures a more accurate sample to gather targeted results in which to draw conclusions and make important decisions.

\begin{testquestion}

\item[1)] What are the benefits of collecting data?
\item[2)]How should data be collected?
\item[3)]Identify issues and/or opportunities for collecting data
\item[4)]Can secondary data be used for both broad and specific uses?
\item[5)]What sources of data should be used to collect information? 
\end{testquestion}
\\
Answers:
	\begin{itemize}
\item[1)] Collecting valuable data is beneficial to proactively address issues and create opportunities.
\item[2)] The two way in which data should be collected is through qualitative and quantitative research. Both approaches, while distinct can produce meaningful data analysis and results.
\item[3)]It is important to conduct an internal and external assessment to understand what are the goals that you want to achieve within your organization, and what are results you are expecting out of the collected data. The opportunities are that you can improve your product or services and satisfy your customers by having more data on likes and dislikes.
\item[4)] Yes, secondary data can be used for both broad and specific uses, since every piece of information and sources is relevant and may be of great use for the researcher. 
\item[5)]The sources used to collect data are pre-existing data, official data, survey data, interviews, and focus groups. 
	\end{itemize}



\begin{glossy}
	\item[Data collection] Is the process of gathering and measuring information on targeted variables in an established system, which then enables one to answer relevant questions and evaluate outcomes. 
	\item[API] Is An application programming interface (API) is a computing interface It defines the kinds of calls or requests that can be made, how to make them, the data formats that should be used, the conventions to follow etc.		
	\item[Organic tweets] Defines the kinds of calls or requests that can be made, how to make them, the data formats that should be used, the conventions to follow, etc.
		
	\item[Retweets] Is a re-posting of a Tweet. Twitter's Retweet feature helps you and others quickly share that Tweet with all of your followers.
	
	\item[Replies] A response to another person's Tweet.
	
	\item[R] Is a free software environment for statistical computing and graphics. It compiles and runs on a wide variety of UNIX platforms, Windows and MacOS.
	
	\item[URL] A colloquially termed a web address, is a reference to a web resource that specifies its location on a computer network and a mechanism for retrieving it. A URL is a specific type of uniform resource identifier.
	
	\item[Clustering] Is the task of grouping a set of objects in such a way that objects in the same group are more similar to each other. 
	\item [Token] The smallest entity that can be subject of a sentiment analysis; it can be an emoticon, a word or an abbreviation.
	\item [Data mining] Is the process of discovering patterns in large data sets.
	
	
\end{glossy}





\input{chapter_Kulkarni}
\input{chapter_Maru}
\input{chapter_Rueda}
\input{chapter_Gokul}

\input{chapter_Bacha}
\input{chapter_Emde}
\input{chapter_Hoffmann}
\input{chapter_Weyers}
\input{chapter_Kolb}

	\chapter{Dictionary-based sentiment analysis}\label{ch:topic1}
	\chapterauthor{Ryan Zidago \\ \href{mailto:zidago.ryan@hs-fresenius.de}{zidago.ryan@hs-fresenius.de}}
	
	\begin{abstract}
		Sentiment analysis has seen a growing interest in the last few years in data science, particularly due to user-generated content (UGC) becoming more ubiquitous than ever on the modern Web 2.0. As such, companies look out for efficient ways to leverage vast amounts of UGC to asses their reputation, as well as improving their current products and services, based on continuous online customer feedback. One way to proceed is by the mean of sentiment analysis, a subset of text mining mostly concerned with sentiments and opinions that are contained in texts. In the following chapter, the reader will be introduced to one of the most practical ways to conduct such an analysis, namely the dictionary-based approach.
	\end{abstract}
	
	\begin{goals}
		\item having a basic understanding of sentiment analysis and its real-world application
		\item understanding the different approaches to sentiment analysis and their trade-offs
		\item knowing how to navigate the open-source ecosystem to find flaws or information in external software dependencies
		\item conducting simple dictionary-based sentiment analysis in an independent manner with the aid of the Python programming language and the vaderSentiment library
		\item critically reflect on the produced work and assess its limitations as well as its possibilities for further improvement
	\end{goals}
	
	\section{Motivation}
	Natural languages (like English, German or French), are highly complex constructs. Humans are efficient at dealing with ambiguity and interpret irony, sarcasm, humor, figure of speech, double meaning, implicitness, innuendos and so forth. Machines are best at following a set of predefined rules, executing repetitive tasks concurrently and at a great pace, during long period of time.
	
	Sentiment analysis is at the crossroad between humans and machines; it leverages both of them to figure out what humans really mean when they voice their opinions, concern or appreciations on the public sphere that is the Internet. On one side, we have humans that built dictionaries full of tokens and their associated sentiment, as interpreted by themselves, and on the other side, we have machines that can compute the sentiment of millions of texts in a comparatively to human, extremely short amount of time.
	
	There are numerous applications to sentiment analysis: it is widely used in marketing, to better understand how consumer describe their experiences online (\citealp{xiang_what_2015}), to asses corporate reputation (\citealp{oconnor_managing_2010}, \citealp{vidya_twitter_2015}, \citealp{chung_evolution_2019}), but also in finance, as a mean to predict stock market movements (\citealp{mohan_stock_2019}) and in health care to understand the current mental health of forum users (\citealp{davcheva_user_2019}).
	
	By the end of this chapter, the reader will have a firm understanding of sentiment analysis, the different approaches used, and how to compute sentiment of textual data, with the intent to extract relevant information for decision making and solving real-world complex business problems.
	
	In the \textbf{Methodological Issues} section, we will be introduced to sentiment analysis, what are sentiments, as well as the different approaches to extract sentiment. In the \textbf{Application in Python} section, we will create a program to compute sentiment out of online travel reviews. In the \textbf{Example from the Real World} section, the reader will be introduced to one of the application of sentiment analysis in the business world: understanding hotel guest experience and satisfaction. Finally, exercises (with their respective solutions) will be made available to the reader.
	\section{Methodological Issues}

	What is sentiment analysis?
	Sentiment analysis is the study of sentiments and opinions that are contained in text. According to \cite{hutto_vader_2014} \citep{hutto_vader_2014}, "sentiment analysis, or opinion mining, is an active area of study in the field of natural language processing that analyzes people's opinions, sentiments, evaluations, attitudes, and emotions via the computational treatment of subjectivity in text".
	
	What is a sentiment? How does it relates to a sentence?
	
	In the context of sentiment analysis, a sentiment can be either positive, negative or neutral.To have a better understanding of how a sentiment relates to a sentence, let us consider the following examples and guess the sentiment they convey:
	\begin{enumerate}
	\item[(1)] "I love this hotel, the staff is always helpful!"
	\item[(2)] "This restaurant mainly offers African food."
	\item[(3)] "I hate this fitness studio, it is always crowded!"
	\end{enumerate}

	Here, in sentence (1), there is a positive meaning, the author states its subjective opinion, carrying a positive sentiment, towards the hotel and its staff. In example (2), the sentence carries a neutral sentiment; it is just a plain objective fact, devoided of subjectivity and assessment, towards what type of food the restaurant offers. Finally, in the last example (3), it become pretty obvious that the sentiment of the text is negative, the fitness studio is describe in bad term "always crowded", and the author states a negative opinion towards the studio "I hate this fitness studio".
	
	What is VADER?
	
	\href{https://github.com/cjhutto/vaderSentiment}{VADER} is a sentiment dictionary. In contrast to a normal dictionary, like the Merriam-Webster, a sentiment dictionary does not map a word to its definition or its meaning, but rather, to its sentiment.
	VADER stands for Valence Aware Dictionary and sEntiment Reasoner, and is purposefully built for social media user-generated content in the form of text (Facebook posts, Reddit comments, Twitter's tweets, etc ...). Another key selling point of the VADER dictionary is its ability to detect the valence of a sentiment: the dictionary registers the intensity of the assessed lexical feature. For instance, the word "good" is evaluated to have a compound score of 0.4404, while the words "great" and "best" have respectively a score of 0.6249 and 0.6369. Here, one can clearly observe that, as the intensity of the word increases, its given compound score also further increases too.\newline
	
	How was the VADER dictionary built? How did they decided which word is worthy to be included in the dictionary and which are not? How did they figure out the exact compound score for each lexical features?
	
	Firstly, the researchers extracted lexical features that were already present in some of the most popular sentiment dictionaries, such as the \href{https://wjh.harvard.edu/~inquirer/}{General Inquirer} or \href{https://csea.phhp.ufl.edu/media.html#bottommedia}{ANEW}. Then, they added other lexical feature that are frequently used in the context of social media text, like emoticons, acronyms and slang. Indeed, one of the selling point of the VADER dictionary is that it is particularly well-suited for social media content. Emoticons too convey sentiment and therefore, are relevant enough to also be included in the dictionary.
	
	Then, they used \href{https://en.wikipedia.org/wiki/Wisdom_of_the_crowd}{wisdom-of-the-crowd} to asses the sentiment of each lexical feature, and only kept those for which there was a large consensus; which resulted in more than 7500 assessed lexical features.
	
	In the next chapter's section, we will get some practical experience by exploring a dataset of \href{https://www.tripadvisor.com/}{TripAdvisor}'s online travel reviews using the dictionary-based approach to sentiment analysis with the vaderSentiment software.
	\section{Application in Python}\label{sec:applicationsinR}
	For this section, we will use 
%	\href{https://www.python.org/}{Python3.8}, 
	Python3.8 (see: \url{https://www.python.org/})
	the latest stable realase currently available at the time of writing, as well as \href{https://github.com/cjhutto/vaderSentiment}{vaderSentiment}, an open-source Python library that provides a simple interface to the VADER sentiment dictionary. Python is a popular general purpose programming language, widely use in web development and scientific computing, including data science. Its syntax is easy to understand and there are plenty of resources freely available on the Web. It is highly recommended that the reader has some familiarity with the Python programming language, as well as with the \href{https://pandas.pydata.org/}{Pandas} package. Those two tools are used a lot in data science, therefore, it is always worth it to invest some time and learn them. For this exercise, there is no need to install either Python or vaderSentiment; a sandboxed environment is provided at \href{https://labs.play-with-docker.com/#}{labs.play-with-docker.com/}. Click on the green \verb#Start# button, then click the \verb# + ADD NEW INSTANCE# on the left panel, and then run the following command in the terminal (the black screen), omitting the \verb#$# sign (a convention to denote that a command should be executed in a terminal): 
	\begin{Verbatim}
$ docker pull ryanzidago/sentiment-analysis
	\end{Verbatim}
	It will download all of the software needed for our hands-on exercise. Once the download is complete, execute the following command:
	\begin{Verbatim}
$ docker run --rm -it ryanzidago/sentiment-analysis
	\end{Verbatim}
	If all went well, you should see a somewhat similar prompt at the terminal (don't worry if the number after the \verb#@# sign is not the same as the one displayed here):
	\begin{Verbatim}
fresenius-student@eb62d59a71bd:~/sentiment-analysis$
	\end{Verbatim}
	
	It looks like you are all set, if not, re-read the different steps carefully, you probably might have missed something trivial. Since we will be using the vaderSentiment Python package, the reader is strongly advised to go through the \href{https://github.com/cjhutto/vaderSentiment#vader-sentiment-analysis}{README} of the library and familiarize themselve with the software. If at any time, you feel lost, you can always fallback to the code for this actual section on \href{https://gist.github.com/ryanzidago/5440bcd66be55ba23ca9b55cf336bab6}{GitHub}.
	
	First, execute the following command to activate the Python shell:
	\begin{Verbatim}
$ python3.8
	\end{Verbatim}
	We will import the \verb|SentimentIntensityAnalyzer| module to compute the sentiment of a text. Type the following lines within the Python interpreter:
	\begin{Verbatim}
from vaderSentiment.vaderSentiment import SentimentIntensityAnalyzer
analyzer = SentimentIntensityAnalyzer()
	\end{Verbatim}
Now that we have correctly imported the module and saved his name to a variable, we can start using its \verb|polarity_scores()| function to infer the sentiment of a sentence:
	\begin{Verbatim}
sentence = """Best breakfast I have ever had, 
but it was way too expensive and the staff was rude and impatient!"""
analyzer.polarity_scores(sentence)
	\end{Verbatim}
	Press \verb|ENTER| and you'll see the following return value of the computation:
\begin{Verbatim}
# => {'neg': 0.276, 'neu': 0.623, 'pos': 0.101, 'compound': -0.6696}
\end{Verbatim}
	vaderSentiment is able to recognizes the different sentiments contained in a sentence. \verb|neg| stands for negative, \verb|neu| for neutral and \verb|pos| for positive. Such as in our example, a sentence can carry various sentiment, even opposite ones. That is why there is a \verb|neg| score of 0.276 and a \verb|pos| of 0.101. The first clause ("Best breakfast I have ever had") conveys a positive sentiment while the last one ("but it was way too expensive and the staff was rude and impatient!") a negative one. If, however, one wishes to know the overall sentiment for the whole sentence, then one should look at the \verb|compound_score|. Again, in our example, the sentence is negative, hence the \verb|compound_score| is negative too. According to the library's \href{https://github.com/cjhutto/vaderSentiment#about-the-scoring}{README}, the following scoring can be adopted to interpret the text's \verb|compound_score|: 
	
	\begin{enumerate}
		\item [] positive: \verb|compound_score| >= 0.05
		\item [] neutral: \verb|compound_score| > -0.05 and \verb|compound_score| < 0.05
		\item [] negative: \verb|compound_score| <= 0.05
	\end{enumerate}

	Another key selling-point of vaderSentiment, is that it is able to factor in the intensity of the expressed sentiment based on punctuation. Replace the exclamation mark in the sentence from the previous example with a dot. You will get a milder score:
	\begin{verbatim}
sentence = """Best breakfast I have ever had, 
but it was way too expensive and the staff was rude and impatient."""
analyzer.polarity_scores(sentence)
# => {'neg': 0.268, 'neu': 0.63, 'pos': 0.102, 'compound': -0.6369}
	\end{verbatim}
	As you can see, the \verb|compound_score| is equal to -0.6369 while with the exclamation mark, it was equal to -0.6696. Feel free to add more exclamation marks and observe how it influences the actual sentence's sentiment intensity. Then, upercase all letter of a high-sentiment word, for example, try the following sentence: 
	\begin{Verbatim}
sentence = """BEST breakfast I have ever had, 
but it was way to EXPENSIVE and the staff was RUDE and IMPATIENT!"""
analyzer.polarity_scores(sentence)
# => {'neg': 0.329, 'neu': 0.566, 'pos': 0.105, 'compound': -0.8087}
	\end{Verbatim}
	Notice how the sentiment's sentence is of greater intensity compared to our previous examples.
	
	Within the sandboxed environment is a csv file containing real-world reviews from TripAdvisor. Let us explore the dataset, mine some data out of it. First, import the panda library then create a dataframe from the csv file, and finally print out the dataframe's columns to familiarize yourself with the dataset:
	\begin{Verbatim}
import pandas
review_dataframe = pandas.read_csv(
'assets/le_meridien_picadilly.csv', 
parse_dates=['created_date', 'stay_date']
)
print(review_dataframe.keys())
# => Index(['_id', 'created_date', 'published_date', 'title', 'stay_date',
'trip_type', 'value_additional_rating', 'location_additional_rating',
'service_additional_rating', 'rooms_additional_rating',
'cleanliness_additional_rating', 'sleep_quality_additional_rating',
'rating', 'text', 'absolute_url', 'helpful_vote', 'photo_counts'],
dtype='object')
	\end{Verbatim}
	To print out one row of the dataset, use the \verb|iloc| function on the dataframe:
	\begin{Verbatim}
first_review = review_dataframe.iloc[0]
print(first_review)
# => _id                                     ObjectId(5e983be6d0014436bd2eb16f)
created_date                                                    2020-04-08
published_date                                                  2020-04-08
title                                                         Disappointed
stay_date                                                       2020-03-31
trip_type                                                             NONE
value_additional_rating                                                NaN
location_additional_rating                                             NaN
service_additional_rating                                                4
rooms_additional_rating                                                  2
cleanliness_additional_rating                                          NaN
sleep_quality_additional_rating                                          3
rating                                                                   3
text                     I was taken here for a birthday treat and we w...
absolute_url             https://www.tripadvisor.com/ShowUserReviews-g1...
helpful_vote                                                           NaN
photo_counts                                                             4
	\end{Verbatim}
	The most important observations here are the \verb|text|, which represents the actual text of the review, as well as the \verb|rating|, which represents the actual score given by the reviewer to the hotel. Let's print out the actual text of the review, and try to infer its sentiment with our intuition alone:
	\begin{Verbatim}
print(first_review['text'])
# => I was taken here for a birthday treat 
and we were very disappointed with the room, 
despite being a Junior Suite 
it was close to being the most depressing room I’ve ever stayed in. 
It was dark and tired 
and we complained and were moved to another room, 
only slightly better, very disappointed. 

The bathroom was tired 
and not sure why the bar of soap was replaced every day, 
seems a waste of money and bad for the environment. 

The air conditioning was not great, 
I would hate to be in the room on a warm day, 
luckily we were there in March so we managed. 

We had access to the Club Lounge which was nice and the staff attentive, 
a nice place to relax after a day out in London. 

Breakfast was nice, 
very extensive but the food started being removed very soon after the end time 
so you need to be down for breakfast well before the cut off time. 

The gym and pool area was quite good for a hotel. 

The younger doorman was very good, 
the older one less so, more interested in gossiping with the staff. 

Great location but that doesn’t compensate for the tired building, 
sadly we will not be returning.
	\end{Verbatim}
	Obviously, this review carries a quite negative sentiment! But what does the vaderSentiment tells us?
	\begin{Verbatim}
analyzer.polarity_scores(first_review['text'])
# => {'neg': 0.144, 'neu': 0.715, 'pos': 0.141, 'compound': 0.3172}
	\end{Verbatim}
	Surprisingly, it informed us, that the review is positive (\verb|compound_score| >= 0.05). Maybe the library is not working as expected? As mentioned in \href{https://github.com/cjhutto/vaderSentiment/blob/master/vaderSentiment/vaderSentiment.py#L600}{vaderSentiment's source code}, the software works best at the sentence level. In the previous example, we fed the whole paragraph to the \verb|polarity_scores()| function. To resolve this issue, we can chunk the paragraph into sentences:
	\begin{Verbatim}
review_text = first_review['text']
sentences = review_text.split(".")
# for every sentence in sentences, compute the polarity_scores()
compound_scores = 
	map(
		lambda sentence: analyzer.polarity_scores(sentence)['compound'], 
		sentences
	   )
	\end{Verbatim}
	Now that we have a list of \verb|compound_scores|, we could sum them up all together, or compute the average \verb|compound_score|. Let's sum them up:
	\begin{Verbatim}
compound_score = sum(compound_scores)
print(compound_score)
# => -1.2965
	\end{Verbatim}
	The computed sentiment is more accurate than before. This was great, but certainly not enough. Let's compute the \verb|compound_scores| for each reviews in the dataset. To do that, we will need to organize our code into functions. The \verb|compute_compound_score()| function takes one argument, the \verb|text| of the review, that is then splitted into sentences. Finally, the function calculcates each \verb|compound_scores| for each \verb|sentences|; sums all \verb|compound_scores| and returns the \verb|compound_score| of the whole review text paragraph:
	\begin{Verbatim}
def compute_compound_score(text):
	sentence = text.split(".")
	compound_scores = 
	map(
		lambda sentence: analyzer.polarity_scores(sentence)['compound'], 
		sentences
		)
	compound_score = sum(compound_scores)
	return compound_score
	\end{Verbatim}
	Now that we have created the \verb|compute_compound_score()| function, we can apply it to the entirety of the dataframe:
	\begin{Verbatim}
compound_scores = 
[compute_compound_score(text) for text in review_dataframe['text']]
review_dataframe['text_compound_score'] = compound_scores
	\end{Verbatim}
	Want to see the result? Try \verb|review_dataframe.iloc[100]| to see the 100th review. At the bottom of the printed message, you will see a new column \verb|text_compound_score| with the newly computed \verb|compound_score| for this review.
	
	This was great. Now, we could categorized each reviews into positive, negative or neutral reviews based on the \verb|compound_score| that we have just obtained. To do that, let us create another function:
	\begin{Verbatim}
def categorizes_based_on_text_compound_score(text):
	compound_score = compute_compound_score(text)
	if compound_score <= -0.05:
		print(f"NEGATIVE: {compound_score} -> {text}\n")
		return 'negative'
	elif compound_score > -0.05 and compound_score < 0.05:
		print(f"NEUTRAL: {compound_score} -> {text}\n")
		return 'neutral'
	elif compound_score >= 0.05:
		print(f"POSITIVE: {compound_score} -> {text}\n")
		return 'positive'
	\end{Verbatim}
	Here, for a given review text, we calculate the \verb|compound_score| with the help of the \verb|compute_compound_score()| function, then, we categorized the returned result as either positive, negative or neutral, based on the scoring annotations mentioned in vaderSentiment's README. As before, we need to apply the \verb|categorizes_based_on_text_compound_score()| function to the dataframe:
	\begin{Verbatim}
categorized_texts =
[categorizes_based_on_text_compound_score(text) 
for text in review_dataframe['text']]
review_dataframe['categorized_based_on_text_compound_score'] = 
categorized_texts
	\end{Verbatim}
	Now, the reader should be able, all by themselve, to inspect the recently updated dataframe to verify if they correctly categorized each reviews.
	
	In the next section, the reader will be introducted to an example of a sentiment analysis project from the real world, where researchers studied the relationship between review rating and sentiment rating.
	
	\section{Example(s) From the Real World}\label{sec:examplerealworld-zidalgo}
	\cite{geetha_relationship_2017} studied the relationship between customer sentiment and online ratings for hotels. They argue that, in order to improve the extremly competitive hotel industry, it is required to improve the understanding of its customers through reviews and ratings that they leave online.
	
	 Their main hypothesis is that customer sentiment polarity has a positive effect on customer rating. \cite{geetha_relationship_2017} used \href{https://en.wikipedia.org/wiki/Simple_random_sample}{Simple Random Sampling} to select 20 budget hotels and 20 premium hotels, all located in the municipality of Goa, India. To classify reviews as either positive, neutral or negative, they also used a dictionary-based approach. Before feeding the sentences to their own implementation of a sentiment software, they processed the reviews by removing punctuations, numbers, \href{https://en.wikipedia.org/wiki/Stop_words}{stopwords}, white spaces. They also converted all letters to lower case and \href{https://en.wikipedia.org/wiki/Stemming}{stemmed} every words in reviews.
	
	As our now experienced reader can notice, their approach has several limitations in comparison with the vaderSentiment library: it does not factor in punctuations, capsulation and emoticons. Hence, "worst ratatouille ever." will carry the same sentiment score as "WORST RATATOUILLE EVER!!! ):" eventhough it is obvious that the latter sentence is more intense than the former one.
	
	They found out that the most frequent words in both categories are quit the same ("hotel", "good", "room", "stay", "staff", etc.). The results of their study showcase a certain consistency between customer ratings and online travel review sentiments: the better the sentiment, the higher the rating. This also prove us that sentiment analysis is effective, otherwise there would be no correlation between review sentiment and review rating.
	
	\cite{geetha_relationship_2017} also discovered that customers from budget hotel were more critical of the hotel where they stayed, with only 55 percent of all reviews having a positive sentiment, while customers from premium hotels were overly satisfied regarding their accommodation, with more than 70 percent positive reviews.
	
	With this valuable information, hotel managers can start to dig deeper to figure out why exactly budget customers aren't as satisfied as premium customers; it could be that budget hotels in Goa need to reasses their communication strategy by making clear from the very begining what the customers should expect. Since budget hotels offer by definition, minimal ameneties and services, hotel managers could probably increase customer satisfaction and loyalty by improving the way hotel staff interacts with its guests.
	
	\section{Conclusion}
	We have learn many things throughout this chapter. We now know that sentiment analysis is about the study of sentiments that are contained within texts. We know that a sentiment is an evaluation of a token, than is categorized as either positive, negative or neutral. We also learned that sentiment analysis is used in a lot of various fields, such as finance, marketing, health-care. We successfuly conducted a small sentiment analysis project, by reading data from a dataset, computing the \verb|compound_score| of some reviews, and categorizing reviews as either positive, negative or neutral. We have learned how to use the vaderSentiment Python library to help us in our analysis.
	
	Obviously, this is not all there is to it. There are many more things to learn. If the reader wishes to further their understanding of the topic, they can freely learn from the following resources:
	\begin{itemize}
		\item \href{https://towardsdatascience.com/scraping-tripadvisor-text-mining-and-sentiment-analysis-for-hotel-reviews-cc4e20aef333}{Web Scraping TripAdvisor, Text Mining and Sentiment Analysis for Hotel Reviews}. This is an online tutorial for conducting a full sentiment-analysis project, from A to Z, from the blog \href{https://towardsdatascience.com/}{TowardsDataScience}, authored by Susan Li. Completing this more extensive and advanced tutorial would be a great step right after having finished reading and working on the current chapter.
		\item \href{http://comp.social.gatech.edu/papers/icwsm14.vader.hutto.pdf}{VADER: A Parsimonous Rule-based Model for Sentiment Analysis for Social Media Text}: This is the actual research paper of the VADER dictionary. The authors explain why did they built VADER and how did they do it, how did they chose to include some words and some not, etc. This paper is very informative, especially regarding how to built a sentiment dictionary.
		\item \href{https://www.cs.uic.edu/~liub/FBS/NLP-handbook-sentiment-analysis.pdf}{Sentiment Analysis and Subjectivity}. This is another research paper, authored by Bing Liu, from the University of Illinois at Chicago. In this paper, the author explain very well what is sentiment analysis and what are the challenges pertaining to it.
	\end{itemize}

	I hope that the reader enjoyed this chapter as much as I enjoyed writting it. Questions, remarks and improvement ideas can be sent to me by email. Next comes some exercises as well as test questions, to assess your newly acquired knowledge.

	\section{Exercises}
	After having successfully graduated from the Hochschule Fresenius, you have goten a position as a junior analyst at a prestigious consulting firm in London. Your manager wants to test you by handing out to you your first big project: helping a struggling hotel to understand their customer: the hotel director, Miss Smith, wishes to know why her customers are satisfied/dissatisfied in order to improve the offered services. She stated that the overall star ratings will not help her better emphatizes with her guests, because those same star ratings provide little context and actionable measures. Instead she wants to know what is hidden in the comment that her guests publish on online booking platforms.
	
	Your task here, is to pretter the work so that other more senior analyst could start to dig deeper in the data. 
	
	Within the sandboxed environment is another dataset containing travel reviews from TripAdvisor, for the Premier Inn London Kensington hotel. Your task is to
	\begin{itemize}
		\item [1.] Find out the percentage (no decimals) of positive and non-negative reviews in the dataset. To proceed, you will need to categorize each reviews as either positive or non-positive reviews, based on the \verb|compound_score| result. This is the stepping-stone for further analysis; this task is crucial.
		\item [2.] Compute the average \verb|compound_score| (3 decimals places) for the entirety of the reviews. She would like to use it as another metrics to assess the quality of the hotel services. 
		\item [3.] Rank the most satisfied \verb|trip_type| (according to the \verb|compound_score|'s average for each traveler categories). She has decided that she would like to better market the hotel based on data-backed customer segmentation.
	\end{itemize}
	
	Answers:
	\begin{itemize}
		\item [1.] 95\% of the hotel guests were satisfied by the hotel itself (consequently, only 5\% were not satisfied).
		\item [2.] The average \verb|compound_score| is equal to 2.077.
		\item [3.] The most satisfied \verb|trip_type| is the \verb|FRIENDS| trip category, with an average \verb|compound_score| equal to 2.209, followed by the \verb|FAMILY| trips with 2.163, then the \verb|COUPLES| with 2.061, and \verb|SOLO| trips with 1.944, and finally the \verb|BUSINESS| category with only 1.729.
	\end{itemize}
	\href{https://gist.github.com/ryanzidago/ef8093cb553913026b0cd3348cf07c7c}{Here} is an example, on how to find out the results.
	
	\begin{testquestion}
		\item [1.] In the context of sentiment analysis, is \textit{anger} a sentiment or an emotion?
		\item [2.] Can a sentence carry several opposite sentiments at the same time?
		\item [3.] What does the \verb|compound_score| returned by the function \verb|polarity_scores()| from vaderSentiment represent?
		\item [4.] If you have a sentence with two clause, the first one being negative, the second one positive, which clause will impact the most the sentence's sentiment?
		\item [5.] Does the vaderSentiment work best at the paragraph or sentence level? Does it give more accurante score after having computed the sentiment of a paragraph or after having computed the sentiment of a sentence?
	\end{testquestion}

		Answers:
	\begin{itemize}
		\item [1.] \textit{Anger} is an emotion. The emotion's sentiment would be \textit{negative}.
		\item [2.] A sentence can carry several opposite sentiments. Let us consider the following example: "The waitress seemed to be under a lot of stress and the pizza arrived quiet late, but it was soooo tasty and worth the wait!". The first clause ("The waitress seemed to be under a lot of stress" and the pizza arrived quiet late") is a neutral/negative statement, while the second clause ("[...] but it was soooo tasty and worth the wait!") carries a positive sentiment.
		\item [3.] The \verb|compound_score| represent the overall sentiment of the sentence. If one would need to select only a single sentiment from the sentence, then one would take a look at the sentiment score contained in the \verb|compound_score|.
		\item [4.] Generally, the last clause of a sentence is the most impactful. You can try to compute the \verb|compound_score| of the example provided in question 2, and then, rearrange the sentence in order to have the last clause at the first position within the sentence, and the first clause at the last (i.e. "The pizza was soooo tasty and worth the wait, but the waitress was seemed to be under a lot of stress and the pizza arrived quiet late.") Observe attentively how much of a difference the position of a clause can make!
		\item [5.] vaderSentiment works best at the sentence level. It can be use to compute the sentiment of a whole document, or just a word, but it will not be as accurate, as when it is done at the sentence level.
	\end{itemize}
	
	\begin{glossy}
		\item[sentiment] In the context of sentiment analysis, a sentiment is an evaluation of a token, that can be either positive or negative. More and more studies also include a neutral sentiment, to have a better granularity in their analysis.
		\item [sentiment analysis] Sentiment analysis is the study of sentiments/opinions that are contained in text.
		\item [dictionary] A database (mostly simply a text file) that maps a token to a score representing the magnitude of its sentiment.
		\item [token] The smallest entity that can be subject of a sentiment analysis; it can be an emoticon, a word or an abbreviation.
		\item [bigram] A pair of tokens conveying meaning together.
		\item [trigram] A set of three tokens conveying meaning together.
		\item [user-generated-content] User-generated content, also more commonly abbreviated UGC, is the content that is published on the web, by users. It could be Facebook posts, Reddit comment, Instagram pictures, Youtube videos and so on. The vast majority of sentiment analysis deals with textual data.
		\item [valence] Valence describes the intensity of a sentiment. Saying that VADER is a valence-aware dictionary means that it is able, not only to determine if a sentiment is either positive, negative or neutral, but also how much positive, negative or neutral the sentiment is, i.e. how intense is the assessed sentiment.
		\item [Web 2.0] the Web 2.0 is the new generation of Web, where the users is not a mere consumer of web-content, but can and is encouraged to contriubte content to the plaftorms that they use (Wikipedia, Facebook, Twitter, TripAdvisor, etc ...).
	\end{glossy}
	

\chapter{This is my title}\label{ch:topic1}
	\chapterauthor{My Name \\ myname@email.de}
	
	\begin{abstract}
		This is some blindtext.
	\end{abstract}
	
	\begin{goals}
		\item This is the first learning objective. 
		\item This is the second learning objective. 
	\end{goals}
	
	\section{Motivation}

This is some blindtext.
Let me cite here someone. \citet{eck2016product} said ...
If you like in brackets like (Huber 2016), then do it like this \citep{eck2016product}.

	\section{Methodological Issues}

This is some blindtext.

	\section{Applications in R}\label{sec:applicationsinR}

This is some blindtext.

	\section{Example(s) From the Real World}\label{sec:examplerealworld}
This is some blindtext.

	\section{Conclusion}
This is some blindtext.

	\section{Exercises}
	\begin{enumerate}[(1)]
		\item This is question?
		\begin{enumerate}[a)]
			\item This is a sub-question.
			\item This is a sub-question.
		\end{enumerate}
	\end{enumerate}
	
	
	\begin{testquestion}
		\item First question.
		\item Second question.
	\end{testquestion}
	
	\begin{glossy}
		\item[word] This is some blindtext.
	\end{glossy}
	


	
	\bibliographystyle{aer}
	\bibliography{lit}
	
\end{document} 
